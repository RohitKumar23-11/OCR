\documentclass[11pt,a4paper]{article}
\usepackage[margin=1in]{geometry}
\usepackage{graphicx}
\usepackage{hyperref}
\usepackage{booktabs}
\usepackage{caption}
\usepackage{float}
\title{PP-OCR End-to-End Study and Training Workflow}
\author{Prepared by: Rohit Kumar (or your name)}
\date{\today}

\begin{document}
\maketitle
\begin{abstract}
This document summarizes the PP-OCR family (PP-OCRv3--v5), dataset selection, conversion steps to PP-OCR formats, reproducible training, and experiments producing detection and recognition results.
\end{abstract}

\section{Architecture overview}
\begin{figure}[H]
\centering
\includegraphics[width=0.9\textwidth]{figures/ppoocr_pipeline.png}
\caption{PP-OCR pipeline (detector $\rightarrow$ cls $\rightarrow$ recognizer).}
\end{figure}
% Add text describing backbones, necks and heads.

\section{Datasets}
\begin{itemize}
  \item COCO-Text V2.0: 63,686 images; 239,506 text instances. (mask annotations) \cite{cocotext}
  \item ICDAR 2019 MLT: multilingual dataset for detection and recognition. \cite{icdar2019}
  \item Others: LSVT, RCTW-17, MTWI, ShopSign.
\end{itemize}

\section{Data conversion and preprocessing}
Add details of conversion scripts and sample snippets.

\section{Training setups and results}
Include tables of hyperparameters and results (precision/recall/f1, accuracy).

\section{Conclusions and next steps}
Summarize trade-offs, multilingual findings, and recommended production models (PP-OCRv5 variants).

\bibliographystyle{plain}
\bibliography{refs}
\end{document}
